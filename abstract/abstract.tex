{
Android rooting has always been a way for tech enthusiasts to explore the hidden power in the android system. It has always been a way for the device owners to customize their devices and run apps that require special privileges. Fundamentally, rooting is all about gaining administrative access to underlying Linux kernel of the Android system, and thus gaining absolute control over the software running on the device. Although rooting is very useful in certain cases, most of the times, it weakens the security of the Android system and open ways for malware to gain privileged access. Google, the company behind Android, has put many security restrictions to stop the practices of rooting, but every time, the tech community was able to bypass the security features of the Android system.

Getting root access on Android versions before Marshmallow was not very difficult. The default method of rooting before Marshmallow was to exploit vulnerabilities in the android systems to mount \code{/system} partition in writable mode(by default, it is only read-only) and then copy the \code{su} binary file, and package to manage user permissions(e.g., SuperUser or SuperSu) and busybox for some common utilities. Then because \code{su} binary has the \code{set-uid} flag set, it can always be run as uid 0 (root).

However, from Marshmallow, things have changed drastically. Google has introduced \code{device-mapper-verity(dm-verity)} to check the integrity of the \code{/system} partition at the boot time. Because all the existing root methods worked by modifying \code{/system} partition in one way or other, the \code{dm-verity} will refuse to boot the system if the \code{/system} is modified even slightly. It does so by comparing the hash value of the \code{/system} partition with some pre-stored hash values. The only way the newer devices can be rooted is to flash a custom \code{boot.img}, and because flashing boot images require an unlocked bootloader, there is currently no method to root devices which ship with locked bootloaders. A locked bootloader will only accept the binaries signed by the manufacturer, and hence it is not possible to alter any file in the Android system with a locked bootloader.

The purpose of this research is to study different rooting methods; the community still uses and the methods that Google has introduced in newer Android versions to stop rooting. The objective of this study is to finally identify some vulnerabilities in the newer Android systems which can be exploited to gain root access.  

\nocite{*}
}
\vspace{5mm}